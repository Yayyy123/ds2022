%!Tex Program = xelatex
%\documentclass[a4paper]{article}
\documentclass[a4paper]{ctexart}
\usepackage{xltxtra}
\usepackage{graphicx}
\usepackage{float}
%\setmainfont[Mapping=tex-text]{AR PL UMing CN:style=Light}
%\setmainfont[Mapping=tex-text]{AR PL UKai CN:style=Book}
%\setmainfont[Mapping=tex-text]{WenQuanYi Zen Hei:style=Regular}
%\setmainfont[Mapping=tex-text]{WenQuanYi Zen Hei Sharp:style=Regular}
%\setmainfont[Mapping=tex-text]{AR PL KaitiM GB:style=Regular} 
%\setmainfont[Mapping=tex-text]{AR PL SungtiL GB:style=Regular} 
%\setmainfont[Mapping=tex-text]{WenQuanYi Zen Hei Mono:style=Regular} 


\title{AVL Tree}
\author{林敬翊\\信计3210300367}
\date{2022/10/28}
\begin{document}
\maketitle
\pagestyle{empty}

\section{设计思路}

因为我们的树是使用AvlTree,所以可以通过自身的调节确保他们平衡以进行查找,所以复杂度为$O(k+logn)$\\

分析,理论上 我们改动了insert函数的情况,每层的时间复杂度为$O(1)$,总的时间复杂度不超过insert自身的时间复杂度。\\

我们最后采用PrintElement操作,并通过打印时间计算时间复杂度。\\
\section{AVL理论}

(1) 查找代价: AVL查找效率最好,最坏情况都是$O(logN)$数量级的。\\

(2) 插入代价:AVL插入操作的代价仍然在$O(logN)$级别上(插入结点需要首先查找插入的位置)。\\

(3) 删除代价:AVL每一次删除操作最多需要$O(logN)$次旋转。因此,删除操作的时间复杂度为$O(logN)+O(logN)=O(2logN)$\\

AVL 效率总结 : 查找的时间复杂度维持在$O(logN)$,不会出现最差情况\\

AVL树在执行每个插入操作时,其时间复杂度在$O(logN)$左右。\\

AVL树在执行删除时代价稍大,执行每个删除操作的时间复杂度需要$O(2logN)$。\\

\section{数值结果分析}
经过运算 我们可以得到下图\\

\begin{figure}[H]
  \includegraphics[width=\linewidth]{avl.png}
  \caption{测试结果}
\end{figure}

由测试结果可以发现 函数在运行的时候是符合一次函数的关系,所以可知函数的时间复杂度收到k控制,为$O(k)$
\end{document}
